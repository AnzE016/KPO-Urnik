\chapter*{Predgovor}
\addcontentsline{toc}{chapter}{Predgovor}


\vspace{1cm}
\indent Dobrodošli pri raziskovanju Oracle Application Express (APEX) – intuitivne in zmogljive razvojne malokodne platforme za ustvarjanje podatkovno vodenih spletnih aplikacij. Ta učbenik je zasnovan z namenom, da vas obogati s kompetencami za izkoriščanje celotnega potenciala Oracle APEX in izdelavo vrhunskih aplikacij za reševanje resničnih poslovnih izzivov.
\vspace{0.5cm}

\noindent Del I v učbeniku je naslovljen ``Kako delate v APEX-u?''. Obravnava osnovne vidike tega orodja. V dvanajstih poglavjih boste potovali po vsebinah, ki vam bodo omogočile razumevanje jedrnih konceptov, pridobitev študijskega razvojnega okolja ter spoznavanje različnih funkcij pri gradnji robustnih aplikacij. Vsako poglavje obravnava specifično temo, zagotavlja jasna navodila ter vabi k preskusu za boljšo učno izkušnjo.

Poglavje 1: ``Kako začnete z Oracle APEX?'' razloži kaj je, za kaj se uporablja in več načinov za pripravo študijskega okolja za učenje in razvijanje aplikacij.

Poglavje 2: ``Kako pripravite bazo podatkov?'' podaja uvod v modeliranje podatkov, upravljanje baze podatkov, manipuliranje podatkov ter poizvedovanje. Za začetnike so predstavljeni vsi pomembno koncepti modeliranja podatkov, ki je obvezna veščina.

Poglavje 3: ``Kako navigirate v APEX-u?'' popelje bralca po funkcijah APEX-a, ki omogočajo razvoj aplikacij, generiranje in prilagajanje različnih spletnih strani in njihovih komponent. Izvajanje in testiranje aplikacije je samo en zavihek oddaljeno od razvojnega okolja.

Poglavje 4: ``Kako izmenjujete podatke v APEX-u?'' daje vpogled v zmogljivosti izvažanja in uvažanja podatkov. Poglavje prikazuje izmenjav z datotekami, preglednicami in tudi s pomočjo storitve RESTful. 

Poglavje 5: ``Kako izdelate prvi osnutek aplikacije?'' vabi k preskusu razvojnih zmogljivosti platforme APEX. Ugotovili boste, da po odločitvi glede vaših podatkov, lahko nemudoma generirate prikupno in uporabno aplikacijo brez programiranja. Razloži vam kako kontrolirano dodeljujete dostopne pravice različnim vlogam uporabnikov s čarovnikom.

Poglavje 6: ``Kako uredite poročila?'' vodi skozi različne vpoglede v podatke od klasičnih poročil do interaktivnih poročil. APEX ima vgrajene funkcije, ki končnemu uporabniku omogočajo prilagajanje brez programiranje ali posredovanja razvijalca.

Poglavje 7: ``Kako uredite obrazce?'' vas uvede v tri splošne tipe spletnih obrazcev, vključno z zahtevnejšimi, ki imajo glavo in podrobnosti. Prilagajanje in generiranje spletnih obrazcev ne bo zahtevalo nobenega programiranja. 

Poglavje 8: ``Kako poročila spremenite v grafikone?'' vas popelje po funkcijah APEX-a, ki omogočajo prikaz podatkov v obliki različnih grafikonov naravnost iz besedilnih poročil.

Poglavje 9: ``Kako urejate menije?'' predstavlja različne navigacijske elemente v APEX-u.

Poglavje 10: ``Kako sodelujete v timu?'' daje vpogled v zmogljivosti APEX-a za podporo dela skupine, kajti zelo redko se zgodi, da na razvoju aplikacije dela zgolj en razvijalec.

Poglavje 11: ``Kako izkoristite galerijo aplikacij in vtičnikov?'' prikazuje moč APEX-a za ponovno uporabo dobrih vzorov.

Poglavje 12: ``Kako upravljate paketne in večjezične aplikacije?'' vas popelje na pot distribucije vaše aplikacije v drugo APEX-ovo okolje z uporabniki, ki govorijo druge jezike.

Del I pokriva tudi vsebine, ki so ključne za varnost aplikacij, strategije namestitev in pripravo za produkcijsko uporabo.

\vspace{0.5cm}
\noindent Del II tega učbenika vas popelje onkraj osnov in predstavi dvanajst privlačnih poslovnih primerov od besednega opisa, preko vseh tehničnih podrobnosti do rešitve - aplikacije. Vsak primer je skrbno dokumentiran, da zagotovi celovito razumevanje razvoja aplikacij z vidika poslovanja, podatkov in uporabniškega vmesnika. Ta del vključuje primere aplikacij za podjetja:
\begin{itemize}
	\item intranetne novice za zaposlene, 
	\item sistem malih inovacij, 
	\item vodenje poslovnih procesov z delovnimi tokovi, 	
	\item kalkulacija materialne kosovnice, 	
	\item sistem za ocenjevanje knjig, 	
	\item vodenje prehrane in diete,	 
	\item razporejanje uradnih ur, 	
	\item zaračunavanje storitev v telekomunikacijskem podjetju, 	
	\item najemanje vozil
\end{itemize}   
za skupine, društva:
\begin{itemize}
	\item katalog rastlin, 
	\item izmenjava rastlin
\end{itemize} 
ter splošno uporabno avtorizacijo in upravljanje uporabnikov.
\noindent V vsakem poslovnem primeru so vključeni:
\begin{itemize}
	\item Poslovni pogled na primer: pregled poslovne situacije.
	\item Definicija problema: iskanje odgovora na kdo in zakaj ima nekdo glavobol.
	\item Primeri uporabe: predstavljene so tri vrste opisov: besedni, polstrukturirani in grafični za pripravo dokumentacije primerov uporabe v UML.
	\item Logični in relacijski podatkovni model: APEX ima vse funkcije za zagon aplikacije iz novih podatkovnih struktur, uporabo in spreminjanje obstoječih, združevanje z drugimi orodji za modeliranje podatkov in podporo za vnaprejšnjega ali obratnega inženiringa. Prizadevanja razvijalcev, da bi zagotovili ustrezne dele podatkov in povezave med njimi ter upoštevanje poslovnih potreb, so osnova za oblikovanje uporabniških vmesnikov.	
	\item Aplikacijski vmesniki: učbenik ponuja HTML strani, poročila, obrazce, polja, menije, gumbe in hiperpovezave, ki materializirajo poslovno situacijo, rešitev poslovnega problema, primere uporabe in podatke z mislijo na končnega uporabnika.
	\item Dodatno učno gradivo: za izboljšanje, pospešitev in pomoč na razvojni poti boste našli povezave do izvoženih paketnih aplikacij, skriptov, podatkov in video vodičev za vsako poglavje. Ti viri vam bodo zagotovili praktične vpoglede, kar vam bo omogočilo, da okrepite svoje znanje in ga neposredno uporabite v projektih iz resničnega sveta.
\end{itemize}

\noindent Ne glede na to, ali ste izkušen razvijalec, ki želi še povečati svoje kompetence, ali začetnik, ki želi raziskati svet APEX-a, je ta učbenik vaš dober vodnik. Upamo, da ga bodo tudi tisti, ki se ne učijo IT, našli kot neprecenljivega spremljevalca na poti do obvladovanja Oracle APEX in gradnje inovativnih aplikacij, ki imajo pozitiven učinek. 

\vspace{0.5cm}
\noindent Učbenik in dodatno študijsko gradivo so zasnovani za približno 75 ur napora študenta (3 ECTS). Upamo, da bodo omogočili različne načine študija:
\begin{itemize}
	\item tečaj ali predmet, pri katerem učitelj predava in izvaja laboratorijske vaje,
	\item mešano učenje (to je del vodi učitelj, del samostojno preštudira študent) in tudi
	\item popolnoma samostojen študij.
\end{itemize}

\noindent Glede na osnovno znanje študentov in razpoložljiv čas za izvajanje tečaja/predmeta lahko učitelji enostavno sestavijo nabor poglavij, ki ustrezajo učnim situacijam, kot so: izvenštudijski tečaji/predmeti, poletne šole, časovno omejeni dogodki za predstavitve malokodnega pristopa za vse študente (ne samo v IT ali računalničarje) in praktično usposabljanje v različnih panogah industrije.

\vspace{0.5cm}
\noindent Za razvoj učbenika in dopolnilnih študijskih gradiv smo uporabljali verziji APEX-a 22 in 23. Avtorji smo prepričani, da bodo razloženi in uporabljeni koncepti ter jedrne tehnologije zelo koristile študentom tudi v prihodnjih verzijah APEX-a.
\newline
\vspace{1cm}
\noindent Uživajte v uporabi čarovnikov in malokodnega programiranja!
\newline

\vspace{1cm}
\noindent red. prof. dr. Robert Leskovar\\
vodja projekta BeeAPEX, predstojnik Katedre za informatiko Fakultete za organizacijske vede UM

